\documentclass{SovJurn/JETPL}
\usepackage{cite} % много ссылок через дефис

\twocolumn

%%% article in English
\lat

\begin{document}
%%% article title
\title{On Quantum and Classical Treatments of Radiative Recombination}
%%% article title - for colontitle (at the top of the page)
\rtitle{recombination}
%%% article title - for table of contents (usualy identical with \title)
\sodtitle{recombination}
%%% author(s) ( + e-mail)
\author{A.~L.~Barabanov$^{1,2}$, K.\,M.\,Belotsky$^{1}$\/, E.~A.~Esipova$^{1}$, D.~S.~Kalashnikov$^{1}$ \, A.~Yu.~Letunov$^{1,2,3}$}

%%% author(s) - for colontitle (at the top of the page)
\rauthor{A.\,L.\,Barabanov, et al.}

%%% author(s) - for table of contents
\sodauthor{Barabanov, Belotsky, Esipova, Kalashnikov, Letunov}
%%% author's address(es)
\address{$^1$National Research Nuclear University MEPhI, Moscow, Russia\\}
\address{$^2$National Research Centre “Kurchatov Institute”, Moscow, Russia\\}
\address{$^3$Russian Federal Nuclear Center – All-Russian Institute of Technical Physics (RFNC-VNIITF), Snezhinsk, Russia}

%%% dates of submition & resubmition (if submitted once, second argument is *)
\dates{\today}

\abstract{The quantum-mechanical solution for the problem of radiative recombination of an electron in the Coulomb field has been known for a long time. However, in astrophysics, the classical approach is sometimes used to treat similar problems. It is shown that the direct use of classical electrodynamics to consider recombination is not adequate due to quantization of the angular momentum. The semiclassical approach, which takes into account such quantization, is proposed, and its consistence with the exact quantum consideration is shown. Possible areas of applicability of the classical approach to problems analogous to radiative recombination are outlined.
The question of which approach is applicable %issue is of special importance  which approach is applicable 
is of particular relevance in cosmological models of self-interacting dark matter, where recombination processes play crucial role in the evolution of large scale structure of the Universe.}
%\textbf{Добавить астрофизики}


\maketitle
\section{Introduction}

The problem of radiative recombination (RR) of an electron in the Coulomb field is thoroughly studied in atomic physics --- see, e.g., the detailed review of Kotelnikov and Milstein \cite{kotelnikov2019electron} with the extensive bibliography and explanations about the significance of this problem for many applications, including those outside of atomic physics. The approximate expression for the recombination cross section was obtained for the first time by Kramers almost a hundred years ago \cite{kramers1923xciii}:
\begin{equation}
\label{1.2}
\sigma^{\rm K}_{n}(\eta)=\frac{32\pi}{3\sqrt{3}}\,\alpha^3a_B^2\,\frac{\eta^4}{n(n^2+\eta^2)}\,.
\end{equation}
Here, $\alpha=e^2/(\hbar c)\simeq 1/137$ is the fine structure constant, $n$ is the principle quantum number, $a_B=\hbar^2/(me^2)$ is the Bohr radius, where $m$ is the mass of the electron, and dimensionless parameter $\eta$ is determined by the following relation
\begin{equation}
\eta=\frac{Ze^2}{\hbar v},
\label{eta}
\end{equation}
\newline
where $Ze$---the charge of the Coulomb center and $v$ is the initial velocity incident electrons. 
\par Summation of cross sections \eqref{1.2} over all $n$ from 1 to infinity gives an approximate expression for the total RR cross section. Replacing summation by integration and some other simplifications introduce additional insignificant errors, but lead to the simple analytical result
\begin{equation}
\label{1.5}
\sigma^{\rm K}_{RR}(\eta)=\frac{32\pi}{3\sqrt{3}}\,\alpha^3a_B^2\,\eta^2\ln (\eta),
\end{equation}
which is called the Kramers formula and widely used for estimations due to its simple form. In the problem of electron capture by protons for the practically significant region, where $50 <\eta <250$, the Kramers formula slightly exceeds the result of accurate quantum calculations consistent with measurements, but with a correction factor $\sim 0.9$ it properly reproduces the dependence of the total RR cross section on the velocity $v$ of incident electrons (see \cite{kotelnikov2019electron} for details). The semiclassical description of inelastic processes corresponds to
\begin{equation}
\eta\gg 1
\label{low_speed_relation}
\end{equation}
%is of particular interest because of the increase in the recombination cross section with decreasing the velocity~$v$.

The problem of radiative recombination can also be formulated and solved in the framework of the classical electrodynamics: an electron scattered by a stationary Coulomb center with a charge of $Ze$ loses part of its energy for radiation and, as a result, passes into a state of finite motion, i.e., is captured by the Coulomb center. The total cross section for such process is obtained by Elutin \cite{elutin}: 
\begin{equation}
\sigma^{\rm Cl}_{RR}(v)=\pi\left(4\pi\right)^{2/5}r^2_0\, Z^{8/5}\left(\frac{c}{v}\right)^{14/5}.
\label{Yelutin}
\end{equation}
Нere $r_0=e^2/(mc^2)$ is the classical electron radius. Note, that the area of applicability for \eqref{Yelutin} differs from \eqref{low_speed_relation} and has the following form: $Z^4(c/v)^2\gg\alpha^{-5}$ \cite{elutin,belotsky2020problems}.

It is easy to see an apparent discrepancy between the equation \eqref{Yelutin} and the expression \eqref{1.5}, rewritten in the form
\begin{equation}
\sigma^{\rm K}_{RR}=\frac{32\pi}{3\sqrt{3}}\alpha r^2_0\left(\frac{Zc}{v}\right)^2 \ln\left(\frac{Ze^2}{\hbar v}\right).
\label{Kramers_inegral}
\end{equation}
\par Firstly, the classical cross section \eqref{Yelutin} has a different dependence on the initial velocity of the electron. %Secondly, prefactors have different orders (mostly due to the absence of the fine structure constant in the classical expression). 
Secondly, prefactors are strongly different mostly due to the absence of the fine structure constant in the classical expression. 
The classical approach to the radiative recombination was previously used in some cosmological problems, where %a slow
Coulomb-like interaction of slow massive particles occurs. Important examples are the work on determination of concentration of magnetic monopoles in the Universe \cite{zeldovich1978concentration} and self-interacting dark matter (DM) \cite{Belotsky_2016, Belotsky:2005dk, Belotsky:2015fuc,Belotsky2017positron,Belotsky:2015rhp,Nazarova:2017xaw}.
Self-interacting DM (SIDM) models become especially popular lately. They may allow solving known problems of standard collisionless cold dark matter scenario of structure formation (like cusp crisis, dwarf galaxies excess predicted, ``too-big-to-fail'' problem) and have physically rich other applications in the searches for DM signals in cosmic rays and underground experiments, explanation of early quasar formation. When Coulomb-like interaction is considered with bound state formation, a quantum approach is usually applied (see, e.g., \cite{Cirelli2017,Petraki2017}) without raising question of its applicability. This also relates to the variety of works considering SIDM models with light mediators. However resulting difference between these two descriptions appears to be of principle \cite{Belotsky_2016}.
Therefore, it seems important to understand how fully quantum and fully classical approaches to radiation recombination are related to each other.
\par It is appropriate to recall here that Kramers' work~\cite{kramers1923xciii}, published in 1923, before the creating of the consistent quantum theory, was carried out within the framework of the semiclassical approach based on the most elementary (fundamental) quantum statements. The main idea of this work was the incorporation of the N.~Bohr correspondence principle into classical expressions. Subsequently, many radiative processes were described in this manner (so called Kramers Electrodynamics) \cite{kogan1992kramer}. The original Kramers' article is based on the circumstance that seems escaped from the researches attention. The careful consideration of the classical RR reveals the fundamental reason of inapplicability of the Elutin's  \cite{elutin} approach: the change in the angular momentum of an electron can not be less than~$\hbar$. It will be shown below that within the straightforward classical treatment the RR angular momentum losses are much less than $\hbar$.  However, combining the classical consideration with the fact of quantizing the angular momentum, we propose a semiclassical approach, in a sense alternative to the one used by Kramers. The total RR cross section obtained within this approach is in agreement with the results of quantum mechanical calculations.
\section{Classical Approach}
Let us consider the classical RR problem in a more general way. The particle with the mass equal to $m$, moves towards the Coulomb center with the initial velocity $v$ and the impact parameter $\rho$. The charges of the Coulomb center and the particle equal to $Ze$ and $-ze$ correspondingly. If one neglects the contribution of radiation in the trajectory formation, than the latter can be considered as hyperbolic (see, e.g, \cite{landau1988theoretical}). The trajectory equation in the polar coordinates $(r,\varphi)$ can written in the following form:
\begin{equation}
r(\varphi)=\frac{p}{1+\varepsilon\cos{\varphi}}\,,
\label{r_phi}
\end{equation}
where $p$ is the orbit parameter and $\varepsilon$ is the eccentricity,
\begin{equation}
p=\frac{L^2}{mzZe^2}\,,\quad 
\varepsilon=\sqrt{1+\frac{2L^2E}{m(zZe^2)^2}}\,,
\label{parameters}
\end{equation}
depend on initial energy $E$ and angular momentum $L$ of the incident particle. The values of these approximate integrals of motion are determined by $v$ и $\rho$: $E=mv^2/2$, $L=m\rho v$. If the direction of the $z$ axis coincides with the vector $L$, than the 
azimuth angle changes from $-\varphi_0$ to $\varphi_0$, where
\begin{equation}
\varphi_0=\arccos\left(-\frac{1}{\varepsilon}\right).
\label{angle}
\end{equation}

In the dipole approximation radiation losses are determined by the well known expressions \cite{landau1975classical}:
\begin{equation}
\label{2.6}
\frac{dE}{dt}=-\frac{2(ze)^2\,{\ddot{\mathbf{r}}}^{\,2}}{3c^3}\,,\quad
\frac{dL}{dt}=-\frac{2(ze)^2[\dot{\mathbf{r}}\times\, \ddot{\mathbf{r}}\,]_z}{3c^3}\,.
\end{equation}
\par Radiation losses $\Delta E$ and $\Delta L$ can be found using the Newton motion law
\begin{equation}
\label{2.7}
m\,\ddot{\!\mathbf{r}}=-\frac{zZe^2}{r^3}\,\mathbf{r}.
\end{equation}
It is convenient to replace time to the the azimuth angle as the integration variable
\begin{equation}
\label{2.4}
L=mr^2\dot\varphi
\quad\Rightarrow\quad
dt=\frac{mr^2}{L}\,d\varphi.
\end{equation}
The total energy radiation loss is determined by the following expression
\begin{equation}
\label{2.9}
|\Delta E|=\frac{4m(z^3Z^2e^5)^2}{3c^3L^5}\,f(\varphi_0),
\end{equation}
where
\begin{equation}
\label{2.9.2}
f(\varphi_0)=\varphi_0\left(1+\frac{1}{2\cos^2\varphi_0}\right)-\frac{3}{2}\tg\varphi_0.
\end{equation}
Similar formulas, but for $z=1$ are presented in the original Kramers' paper \cite{kramers1923xciii}.

Despite the fact that the decrease in energy $|\Delta E|$ is small relative to the initial energy, it fully determines radiative recombination. Radiation losses is the reason of the transition from infinite to finite area of motion. Therefore, for a given initial velocity $v$ the effect is the stronger, the more significant the value \eqref {2.9}, i.e. the smaller the angular momentum $L$ and, hence, the impact parameter $\rho $. However, there are additional limitations for $L$ ($\rho$) that comes from low values of angular momenta. Equations \eqref{2.6} are valid in all parts of the trajectory, where the dipole approximation is applicable: the initial velocity is much smaller than the speed of light. It is easy to derive the maximum value of speed
\begin{equation}
v_{\rm max}=v\sqrt{\frac{\varepsilon+1}{\varepsilon-1}}\,.
\label{vmax}
\end{equation}
The incident particle's velocity equals to $v_{\rm max}$ when $\varphi=0$. This circumstance and the dipole approximation condition $v_{\rm max}\ll c$ lead to the following limitation for the angular momentum
\begin{equation}
L=m\rho v\gg\frac{(1+\varepsilon)zZe^2}{c}\,.
\label{2.4.4}
\end{equation}
\par It follows from the relation \eqref{2.4.4} that trajectories with low values of the impact parameter are not taken into account. However, as it will be shown below, it does not affect the recombination process.
\par For angular momenta values, which simultaneously satisfies \label{2.4.4} and small enough to provide sufficient energy losses $|\Delta E|$ for the transition into the area of finite motion, one obtains: $\varepsilon\to 1$ и $\varphi_0\to \pi$. So, using \eqref{2.9} and \eqref{2.9.2} radiation loss for the energy is equal to
\begin{equation}
\label{EnergyLossFromLL}
|\Delta E|=\frac{2\pi m(z^3Z^2e^5)^2}{c^3L^5}\,,
\end{equation}
see analogous Eq.(23) in \cite{kramers1923xciii} and Eq.(7) in \cite{elutin}, obtained for $z=1$.
\par Note, that a particle, which initially has energy $E=mv^2/2>0$ turns into state with negative value of $E$. 
\begin{equation}
\label{2.11.1}
|\Delta E|>E.
\end{equation}
This condition can be written as follows 
\begin{equation}
\label{2.11.2}
(4\pi)^{2/5}(z^3Z^2)^{4/5}r_0^2\left(\frac{c}{v}\right)^{14/5}>\rho^2\,.
\end{equation}
The left-hand side of the relation \eqref{2.11.2} can be denoted as $\rho_m^2$. It means that the condition of the transition into the bound state (for the Coulomb potential) can be formulated as: $\rho<\rho_m$. Thus, the classical RR cross section is equal to $\pi\rho_m^2$. Substitution of $z=1$ leads to the expression \eqref{Yelutin}. Note, that the condition \eqref{2.4.4} turns into  $v\ll c\sqrt{z/Z}$, which is valid for moderate values of $z$ and $Z$ (incident particle in this treatment are non-relativistic). Let us now consider particles with such a small impact parameters so the condition \eqref{2.4.4} is not satisfied. There particles achieve relativistic velocities near the Coulomb center. However, radiation losses for such electrons are much greater than $|\Delta E|$ for non-relativistic particles. So, the condition \eqref{2.11.1} is satisfied for these incident particles and they turn into the area of finite motion.
\par Note that the particle's energy changes from the positive value $E=mv^2/2>0$ to a negative one. It might be confused, because we don't take into account energy loss effect on the trajectory. Nevertheless, it doesn't affect the incident particle motion. Indeed, the main contribution to the radiation process is made by motion on a small segment of the trajectory (see the rotational approximation in \cite{kogan1992kramers}. Thus, despite the 
deviation of the eccentricity from the unit in both directions, which results in transition of the trajectory from hyperbolic to elongated elliptical, the motion regime near the Coulomb center remains practically unchanged.
 
\section{Semiclassical Approach}
\par Within the classical approach described above, each particle falling with a given initial velocity $v$ and an impact parameter $\rho$ emits a strictly defined energy $|\Delta E|$, and goes into a state of finite motion if this energy satisfies the condition \eqref{2.11.1}. The situation is different in the semiclassical approach. Here, the particle only with a certain probability $P(\rho)$ passes into a bound state (hereinafter, we assume the velocity $v$ to be fixed, but trace the dependence of the quantities on $\rho$). The corresponding semiclassical cross section is equal to
\begin{equation}
\label{crosssection}
\sigma=2\pi\int\limits_{0}^{\infty}P(\rho)\,\rho d\rho.
\end{equation}
\par The semiclassical Kramers' approach is based on the spectral expansion of the energy radiation loss
\begin{equation}
\label{spectrum}
|\Delta E(\rho)|=\int\limits_{0}^{\infty}I(\rho,\omega)d\omega.
\end{equation}
In the framework of the quantum theory, the energy flux emitted by electrons (with given $v$ and $\rho$) in the frequency range from $\omega$ to $\omega+d\omega$ is provided by photons with energy $\hbar\omega$. If we assume that an electron with probability $q (\rho,\omega)d\omega$ emits a photon with energy $\hbar\omega$, then according to the correspondence principle it is necessary to equate the total energy $NI(\rho,\omega)d\omega$, emitted by $N$ electrons in the frequency range from $\omega$ to $\omega+d\omega $ in the framework of the classical theory, and the total energy $\hbar\omega Nq(\rho,\omega)d\omega\,$ emitted by the same $N$ electrons in the same frequency range in the framework of the semiclassical approach. These manipulations leads to the following relation
\begin{equation}
\label{difprobability}
q(\rho,\omega)=\frac{I(\rho,\omega)}{\hbar\omega}\,.
\end{equation}
\par The integral, calculated over all frequencies within the classical approach represents the full probability of the photon emission \footnote{Note, that the expression \eqref{fprobability} is absent in the Kramers' original work \cite{kramers1923xciii}. The quantity in the right-hand side of the equation \eqref{fprobability} is much less than unit, which means that a big number of electrons do not emit. It seems, that such a conclusion looks too extraordinary for the paper in 1923. See the footnote 5 below.}
\begin{equation}
\label{fprobability}
P^K(\rho)=\int\limits_{\Delta_n\omega}q(\rho,\omega)d\omega=\dfrac{8}{\sqrt{3}}\alpha^3\bigg(\dfrac{Z\hbar}{L}\bigg)^2.
\end{equation}
\par Matching every bound state $n$ with the finite frequency interval $\Delta_n\omega$ allows one to obtain the differential transition probability
\begin{equation}
P_n^{K}(\rho)=\int\limits_{\Delta_n\omega}q(\rho,\omega)d\pmega.
\label{nprobability}
\end{equation}
\par Evaluating the integral \eqref{crosssection} with these probabilities leads to the formula \eqref {1.2}.%However, within such a procedure, there is no idea whether there is an area of applicability of the full classical expression \eqref{Yelutin} for the total RR cross section.
The Kramers' approach let one 
\par That is why we return to the classical approach and note that the loss of energy \eqref{EnergyLossFromLL}, which is necessary for binding the system, is inevitably accompanied by a loss of angular momentum. Thus, we consider this loss, which is determined according \eqref{2.6} and \eqref{2.7} by the following relation:
\begin{equation}
\frac{dL}{dt}=-\dfrac{2z^3Ze^4}{3m^2c^3r^3}L.
\label{brem_an_mom}
\end{equation}
Using \eqref{2.4} to replace the time $t$ by the azimuth angle~$\varphi$ as well as expressions \eqref{r_phi}, \eqref{parameters} and the result $\varphi_0\simeq\pi$, for the loss of angular momentum for the entire motion of the incident particle, one obtains
\begin{equation}
|\Delta L|=\frac{4\pi (z^2Ze^3)^2}{3L^2c^3}
\label{ddl}
\end{equation}
with additional assumption $|\Delta L|\ll L$, which validity is ensured by the relation \eqref{2.4.4}.

Then we take the fundamental fact of discreteness of angular momentum
\begin{equation}
L=l\hbar,
\label{q_ang_mom}
\end{equation}
where $l$ is an integer number\footnote{Obviously, the spin plays no role in this approach}. It means that \eqref{ddl} can be rewritten in the form
\begin{equation}
|\Delta L|=\frac{4\pi (z^2Z)^2}{3l^2}\left(\frac{e^2}{\hbar c}\right)^3\hbar.
\label{ddl2}
\end{equation}
It is easy to see from \eqref{ddl2} that $\Delta L \ll \hbar$, if $z$ and $Z$ are not too high. This is the reason why the completely classical approach is not applicable to the problem of radiative capture description in such an elementary system as an electron and a positively charged nucleus. However, it is possible to construct the semiclassical approach to provide a solution to this problem. For the sake of simplicity, we assume $z=1$.
Let us make an assumption that an electron can emit only with the probability equal to $P(\rho)$. Moreover, the value of emitted angular momentum equals to $\hbar$. Then, according to the correspondence principle, one should equate angular momentum $N|\Delta L|$, emitted by $N$ electrons within classical treatment, and the same value derived using the semiclassical approach $NP(\rho)\,\hbar$  
\begin{equation}
P(\rho)=\frac{|\Delta L|}{\hbar}=
\frac{4\pi(Ze^3)^2}{3\hbar m^2\rho^2v^2c^3}\,.
\label{probability2}
\end{equation}
One can repeat the same manipulations with the energy radiations losses. If the semiclassical emitted energy equals $NP(\rho)\Delta E^{\prime}|$, then one obtains the following relation
\begin{equation}
|\Delta E^{\prime}|=\frac{|\Delta E|}{P(\rho)}=
\frac{3\hbar (Ze^2)^2}{2m^2\rho^3v^3}\,.
\label{energy2}
\end{equation}
\par Substitution of \eqref{energy2} into the left-hand side of \eqref{2.11.1} instead of $|\Delta E|$, leads to the $\rho<\rho_{\rm max}$ instead of \eqref{2.11.2}, where
\begin{equation}
\rho_{\rm max}=\frac{1}{m}\left(\frac{3\hbar(Ze^2)^2}{v^5}\right)^{1/3}.
\label{rhomax}
\end{equation}
\par The value $\rho_{max}$ is the maximum impact parameter, which an electron is captured into one of the bound states in the semiclassical approach.
\par Thus, the total RR cross section is determined by the integral \eqref{crosssection} with the probability $P(\rho)$ \eqref{probability2} and limits from some minimal impact parameter value $\rho_{min}$ and \eqref{rhomax}. If one consider
\begin{equation}
\rho_{\rm min}=\frac{\hbar}{mv}\,,
\label{rhomin}
\end{equation}
taking $\hbar$ as a minimum emitted angular momentum, we obtain the expression for integral from \eqref{crosssection}:
\begin{equation}
\int\limits_{\rho_{\rm min}}^{\rho_{\rm max}}\frac{d\rho^2}{\rho^2}=
2\ln\left(\frac{\rho_{\rm max}}{\rho_{\rm min}}\right)=
2\left(\frac{1}{2}\ln 3+\frac{2}{3}\ln\eta\right).
\label{integral}
\end{equation}
\par Taking into account \eqref{low_speed_relation}, one can neglect the first term in \eqref{integral}. As a result, in this semiclassical approach, we obtain the following expression for the total RR cross section
\begin{equation}
\sigma_r=\frac{16\pi^2}{9}\alpha^3a_0^2\, \eta^2\ln (\eta).
\label{final_sigma}
\end{equation}
The only difference of this expression with the Kramers' formula \eqref{1.5} is slightly different prefactor. Notice that the ratio of this prefactor to that from \eqref{1.5} is approximately~$0.9$, that is why the obtained result is closer to the accurate quantum result, as it was outlined in the Introduction.

\par To conclude this section we would like to underline the important circumstance. The incident electron can emit angular momentum greater than $\hbar$ for trajectories with small impact parameter. According to \eqref{ddl2} it might happen when
\begin{equation}
z^2Z>l\, \sqrt{\frac{3}{4\pi}} \left(\frac{\hbar c}{e^2}\right)^{3/2}\simeq 10^3 l.
\label{final}
\end{equation}
In this case, apparently, classical treatment provides the description of a contribution to RR cross section. In problems of dark matter, the condition written out will be softer if the analog of the fine structure constant turns out to be not so small as the number 1/137. For example, for magnetic monopoles the charge $q_M\sim c\hbar/e$.

\section{Conclusion}
The classical description of the radiative recombination within the dipole approximation leads to the apparent discrepancy with the known quantum result. The main reason of it is that the electron angular momentum loss is much less than~$\hbar$. However, for the dipole transitions the orbital quantum number has to change: $\Delta l=\pm 1$. This quantum condition turns out to forbid photon emission by an electron in the Coulomb field within the straightforward classical consideration.
In order to overcome this problem the alternative semiclassical calculation is presented. Such approach leads to the correct expression for the recombination cross section. The issue of the applicability of the classical or quantum approaches had been earlier found to be especially relevant in cosmological problems related to the self-interacting dark matter.
\section{Acknowledgement}

We would like to thank S.G.Rubin for interest to this work with useful discussion. The work of K.M.B. (on connection of this task with dark matter problem) was supported by Ministry of Science and Higher Education of the Russian Federation by project No 0723-2020-0040 “Fundamental problems of cosmic rays and dark matter”. \textbf{The work of E.A.E. (on calculation of applicability conditions ...) was supported by ... Basis }

% \begin{thebibliography}{99}

% \bibitem{kotelnikov2019electron}
% I.~A. Kotelnikov and A.~I. Milstein.
% \newblock Electron radiative recombination with a hydrogen-like ion.
% \newblock {\em Physica Scripta}, 94(5):055403, 2019.

% \bibitem{kramers1923xciii}
% H.~Kramers.
% \newblock Xciii. on the theory of x-ray absorption and of the continuous x-ray
%   spectrum.
% \newblock {\em The London, Edinburgh, and Dublin Philosophical Magazine and
%   Journal of Science}, 46(275):836--871, 1923.

% \bibitem{elutin}
% P.~V. Yelutin.
% \newblock Classical cross-section for recombination.
% \newblock {\em Theoretical and mathematical physics}, 34(2):180--184, 1978 (in Russian).

% \bibitem{zeldovich1978concentration}
% Ya.~B. Zeldovich and M.~Yu. Khlopov.
% \newblock On the concentration of relic magnetic monopoles in the universe.
% \newblock {\em Physics Letters B}, 79(3):239--241, 1978.

% \bibitem{Belotsky_2016}
% K.~M. Belotsky, E.~A. Esipova, and A.~A. Kirillov.
% \newblock On the classical description of the recombination of dark matter
%   particles with a coulomb-like interaction.
% \newblock {\em Physics Letters B}, 761:81–86, Oct 2016.

% \bibitem{Belotsky:2005dk}
% K.~M. Belotsky, M.~Yu. Khlopov, S.~V. Legonkov, and K.~I. Shibaev.
% \newblock {Effects of new long-range interaction: Recombination of relic heavy
%   neutrinos and antineutrinos}.
% \newblock {\em Grav. Cosmol.}, 11:27--33, 2005.

% \bibitem{Belotsky:2015fuc}
% K.~M. Belotsky, E.~A. Esipova, and A.~A. Kirillov.
% \newblock {On Recombination of Dark Matter Particles with Dark U(1)
%   Interaction}.
% \newblock {\em Phys. Procedia}, 74:24--27, 2015.

% \bibitem{Belotsky2017positron}
% K.~M. Belotsky, E.~A. Esipova, and A.~A. Kirillov.
% \newblock Constraints on the model of dark matter with coulomb-like interaction
%   explaining positron anomaly.
% \newblock {\em Journal of Physics: Conference Series}, 934:012020, 12 2017.

% \bibitem{Belotsky:2015rhp}
% K.~M. Belotsky, E.~A. Esipova, M.~Yu. Khlopov, and M.~N. Laletin.
% \newblock {Dark Coulomb binding of heavy neutrinos of fourth family}.
% \newblock {\em Int. J. Mod. Phys. D}, 24(13):1545008, 2015.

% \bibitem{Nazarova:2017xaw}
% N.~O. Nazarova and K.~M. Belotsky.
% \newblock {Estimation of the Density of the Self-Interacting Dark-Matter
%   Component within a Clump with Allowance for Recombination}.
% \newblock {\em Phys. Atom. Nucl.}, 80(6):1177--1180, 2017.

% \bibitem{landau1988theoretical}
% L.~D. Landau and E.~M. Lifshits.
% \newblock Theoretical physics, vol. 1: Mechanics.
% \newblock {\em English translation, Pergamon Press, Book p}, 9099:35, 1988.

% \bibitem{landau1975classical}
% L.~D. Landau and E.~M. Lifshitz.
% \newblock Classical field theory.
% \newblock {\em Course of Theoretical Physics}, 2, 1975.

% \bibitem{KoganPlanckConstant}
% V.~I. Kogan.
% \newblock The discovery of the planck constant:
%   ''roentgenoscopy'' of the scientific
%   situation (1900). missed opportunities in the choice of the second step (on
%   the centenary of the first step of quantum theory).
% \newblock 43(12):1253--1259, dec 2000.

% \end{thebibliography}
\bibliographystyle{unsrt}
\bibliography{bibliography/bibl}
\end{document}